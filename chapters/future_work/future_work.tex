\chapter{Outlook}\label{outlook}

This thesis has covered many different aspects of PVC biology, and also represents a `revival' of a project that had lain idle for a number of years. Consequently, there are many avenues which have generated preliminary data which could be built on significantly in future.

As far as Chapter \ref{structbioinfo} is concerned, the process could be iterated essentially \emph{ad infinitum}, each time benefitting from better homologies through new depositions in the various databanks. In the 3-4 years of this work alone, new homologies to almost every protein have been found. Of course, a key advancement will be the final experimental elucidation of a full PVC structure, hopefully in the not too distant future, as the collaboration with the Max-Planck in Dortmund continues. Once this is done, it would be worthwhile repeating the process for many, if not all PVC proteins, to obtain more accurate homology models for future use. The models obtained in this study will prove useful many aspects of future lab work potentially; providing a reference for the effects of, for instance, mutagenesis, or for the design of anti-peptide antibodies against elements expected to be available on the surface of the complex.

For Chapter \ref{bioinformatics}, it would be interesting to repeat the workflow with a greater number of PVC loci from a greater diversity of genomes. While not reported here, progress has been made on automatically detecting and `isolating' PVC sequences from other genomes, based in part on the criteria defined in this chapter. It would also be advantageous to attempt to fully automate the process of syntenic orthologue clustering and matrix creation for calculation of the Adjusted Wallace Coefficient - at present these tasks are more subjective than is ideal. Hand in hand with this, as ever, it would also be useful to sequence even more genomes for further study.

As the first experimental chapter, Chapter \ref{tailfibres} has opened up a great many future avenues for study. Setting aside the practicalities for a moment, the most obvious tasks are to try and repeat the studies here for as many putative PVC tail fibres as possible. That includes binding partner characterisation (both proteomic and glycan array-based). Some progress along this path is underway, as the ``Pnf" fibre that was cloned in this work, but slightly more recalcitrant to work with, has only recently been subjected to the same affinity experiments as the ``Lumt" fibre. Since then, a further tail fibre from the ``Unit 4" operon of \Pasy{} TT01 has been cloned (as this is the same PVC which is currently under study in Dortmund). Just counting the PVCs which have been studied from the 3 genomes focussed on in this PhD, that leaves approximately another 13 tail fibre proteins which could be cloned and assayed. Much easier said than done, of course, would be further attempts to crystallise and finally resolve the structures of these proteins once and for all, instead of relying on indirect measurements like circular dichroism and SDS-PAGE. Exciting future work will be to recapitulate similar studies of engineering tail fibres, with the intent of altering the specificity of the PVCs, such that they may be able to convey their cargoes to cell/tissue types of our choosing. Current experiments ongoing in the lab are swapping the fibres between different PVC structures, to examine any effects on assembly or binding/function (it's possible for example, that they won't be able to transduce a contraction signal). It is not known for certain that the contraction signal is conveyed through the tail fibres, though this is the case for the T4 phage, so identifying if the fibres are implicated in this function will also be well worth testing in future. If they remain functional with `transplanted' fibres however, this becomes a tool for exploring the role of different PVCs and their natural specificities for different cells.

Time was limited for much of the reporter microscopy work undertaken in the final chapter, which leaves much left to do. The natural progressions for this study are to, firstly, perform the microscopy assays under as wide a variety of potential inducing conditions as possible, to see there are differential responses in the deployment of the PVCs - so far, only standard lab rich media conditions have been tested (albeit over a long growth cycle). Secondly, to obtain more quantitative data about the population heterogeneity, these reporters should be subjected to flow cytometry or flow-assisted cell sorting. The FC/FACS process can therefore also extend all the different conditions that would be tested. Sequencing of sub-populations isolated by FACS could potentially offer insights in to the genomic/transcriptomic state of the proposed `sacrificial populations', which would not only be enlightening in terms of PVC biology, but might also have information to offer for understanding the link between a sacrificial population and association with the nematode, or other fundamental aspects of its life cycle. On the subject of transcriptomics, an interesting experiment would be direct RNA sequencing, now made possible by the likes of Oxford Nanopore. This would shed further light on the potential role of transcript elongation in the PVC operons. The current theory is that the PVCs are produced progressively extending transcripts (to produce large amounts of the 5' genes) or from one long transcript which is potentially re-used.

Finally, it will no doubt be necessary to continue grinding away at the heterologous expression of as many PVC operons as possible. It seems the most promising way to do this is still `the old fashioned way', and some successes so far, both in terms of their construction, but also their functionality, provide confidence that this is the most viable way to proceed.


In summary, `the PVC project' is nothing short of enormous. There is a great deal that can be done to extend this thesis, and a great deal more besides, that this thesis hasn't even touched on. The project will no doubt yield many more PhDs-worth of data and insight to come.

\vspace{1cm}

{\hfill{\Huge\adfclosedflourishleft}\hfill}



