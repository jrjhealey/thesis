\chapter{Discussion}\label{Discussion}
\pagestyle{IHA-fancy-style}

As exemplified by \cite{Sarris2014} and \cite{Hurst2004}, protein-translocating, phage-like macromolecular complexes are increasingly widespread, with many examples identified to date, including the Metamorphosis Associated Complex of \emph{Pseudoalteromonas luteoviolacea}, the Antifeeding prophage(s) of \emph{Serratia entomophila}, the Type 6 (and other secretion systems) found in many genera, and the PVCs themselves. The Afps and PVCs are somewhat different in that they are released from the host cells as individual `needles', and capable of exerting their effects `at range'. \emph{Photorhabdus} and the PVCs remain more unique still for being the only examples identified to date where multiple variants are encoded within a single genome. What's more, while so far the Afps have only been observed on plasmids, PVCs have only even been found chromosomally integrated.

As a highly effective insect pathogen, \emph{Photorhabdus} deploys the PVCs as just a small, but potentially highly versatile, part of its arsenal, facilitating pathogenicity and possibly other aspects of its life cycle. Given the bacteria's unusual life cycle as both a pathogen and a mutualist, it has to be capable of virulence against assorted insect prey, `self-defence' to ensure the protection of the killed prey in the soil, all the while retaining symbiosis capability, to ensure continued re-association with the nematode. Consequently, \emph{Photorhabdus} requires such a large repertoire of molecules, so it can use to influence the biology of a great many other organisms. Preliminary data in the lab has also begun to suggest that the PVCs may have roles beyond purely virulence, and that one of the PVCs may be responsible in part for the association between the bacteria and the nematode host.

This thesis has attempted to take a slightly different tack than many of the publications to date. Specifically, works like the studies of \cite{Sarris2014}, and even the original PVC discovery paper by \cite{Yang2006}, have tried to contextualise the PVCs in the pantheon of known caudate structures. Certainly, to begin to understand the basics of the structures this is a necessary step, and elements of the thesis continue in this vein. But, by leaning on this existing knowledge, a more novel approach of comparing and contrasting the variability between different PVC operons from various genomes, offers a completely different way of analysing these remarkable biological entities.

In summary, this work has examined the nuanced differences between PVC operons, both experimentally and computationally identifying various `shades' of PVCs, not only in terms of their structural and payload differences, but also potentially in their regulation and deployment. Below some of the key insights from this work are summarised below, from each chapter discussion, and the relevant future study.

\section{Chapter \ref{structbioinfo}: New insights on PVC assembly and structure}
\subsection{A new role for an inner sheath paralogue}
Subtle differences in the obtained structural models suggest that one of the inner sheath components, likely PVC5, may not actually form part of the inner sheath \emph{per se}, instead forming an interfacing collar, similar to that of gp48/54 in the T4 phage. To date, it has been assumed that PVC1-5 simply contribute to the tube proper, and stoichiometric necessity was the given hypothesis. It seems that this may not be the case, when the electrostatics of PVC5 are examined, and the observation is made that it is translationally coupled to the rest of the spike/baseplate cluster of genes. If this is the case, this would also cast doubts on the contributions of the exterior sheath proteins, potentially implicating one or more of them in other roles such as other baseplate interfacing `adapter` proteins. Evidence in the case of the outer sheath proteins is sparser however, as little difference could be identified in their simulated structures.

\subsection{PVCs have profoundly negatively charged surfaces}
Electrostatic observations of the exterior aspect of the putative outer sheath proteins reveals an overwhelming negative charge. This is borne out by subsequent experimental observation of their ability to bind to quaternary amine-based ion exchange columns. An extremely attractive hypothesis for this is that it may represent an `evolved' surface, minimising the effects of trypsinisation and endocytosis, therefore potentially increasing the longevity of circulating released PVCs in an infection scenario \citep{DelTordello2016, Kaur2012}.

\subsection{PVCs resemble `hybrid' caudate structures}
While not strictly a new observation, since it has been known for some time that PVCs resemble elements of both phage and T6SSs, \vref{structbioinfo} has demonstrated conclusively that the PVCs are reminiscent of a patchwork of different structures. Their outer sheaths, for instance, are highly similar to those of the R-type pyocin, yet their inner sheaths are not (or at least less so, having more in common with the T4 gp19 ortholog). The R-type pyocins, in turn, are thought to be descended from the P2 phage, rather than T4. Similarly, the models obtained for the VgrG-like spike complex appear `stripped down' and simpler, much more like that of the T6SS than the T4 phage. Now, this is not to suggest that \emph{Photorhabdus} has co-opted different components from different sources, rather it probably represents a honing and convergent evolution to handle the subtly different functions (translocating protein versus DNA for instance). Some questions remain however, for instance, if the spike complex is more similar to T6SS, but the sheath is more similar to non-T6SS structures, is there really the need for collar adapter proteins \citep{Renault2018}. In short, no single orthologous structure dominates the `homology signal' detected in querying PVC proteins.

\subsection{Structural evidence for PVC14 as a `tape measure protein'}
While the models are likely far from accurate, and resolving the structures of tape measure proteins \emph{in vitro} is an obstinate challenge, there is useful information in the models obtained. The work of \cite{Rybakova2015} on the Afp equivalent locus provides some compelling evidence for their role as tape measure proteins, controlling the polymerisation of the Afp/PVC tubes. Now this \emph{in silico} work exemplifies the striking helical secondary structure, and characteristic hydropathy of the locus, adding weight to the hypothesis.

\subsection{Identification of possible new roles for certain loci}
Structural modelling has revealed the striking structural similarity of PVC9 to a tube initiator protein, forming part of the baseplate complex, and revealing some spurious annotations attributed to certain protein structures within the PDB. Some preliminary evidence from HMM searches and backed up with preliminary modelling work has also identified PVC10 as a remote orthologue, on the very limits of sequence-based detection, to PAAR-spike tip proteins. Despite not containing the eponymous Pro-Ala-Ala repeats, potentially conserved metal binding residues were observed, along with characteristic secondary structure profiles. This has helped to further unpick some of the remaining `dark matter' of the PVCs.

\section{Chapter \ref{bioinformatics}: Understanding PVC variability \& mobility} 
\subsection{PVCs as highly variable operons}
Performing gene-by-gene phylogenetics within the PVC operons necessitated that the genes be clustered by synteny and orthology as a first step. This process alone reveals that the PVC operons are home to significant differences. There are, for example, numerous gene deletions in several of the operons, and curating the ``Lumt" operon revealed that its `operon core' (the last, approximately, six genes), are substantially different (and additional). Once the analysis proper had been completed, it showed that the putative tail fibre proteins, PVC13, are by far the most variable and incongruent genes in the whole operon. It was also revealed that both the ``Lumt" and ``Pnf" operons are frequent outliers, being somewhat sequentially different from the rest of the PVCs (and from each other). This has some potential consequences for the use of the ``Pnf" operon as the lab model, however its function and structure remain among the better characterised, so this is unlikely to change. A good candidate for an alternative model operon would be the ``Cif" operons. Firstly, they would make good/better models as they are present in the \Plum{} genomes as well as the \Pasy{} ones (unlike ``Pnf"), and frequently cluster together, with the orthologue from \Plum{} as the relative outgrouping which is to be expected. The topologies for many of the other PVCs change within the trees, and this is especially true for ``Pnf" and ``Lumt".

\subsection{PVCs are likely older than first thought}
This workflow revealed a surprising conclusion. Inspecting the PVC sequences unequivocally reveals the presence of many tandem repeats, insertion elements, transposons, and other paraphernalia associated with horizontal gene transfer. However, the fact that the PVC genes are by-and-large congruent with the consensus tree and the known species topology, suggests that the PVC operons have been `co-speciating' with their host genomes for some time. The hallmarks of horizontal gene transfer seen in the genomes probably speak to the mechanism of initial acquisition of the sequences. Certainly, the ``Unit1" to ``Unit4" operons in \Plum{} which are tandem to one another are proposed to have been carried in by the integration of a large plasmid \citep{Yang2006}.

\section{Chapter \ref{tailfibres}: PVCs have hyper-variable, chimeric tail fibres}
\subsection{PVC tail fibre proteins share hallmarks of real tail fibre proteins}
In Chapter \ref{tailfibres}, the first experimental confirmation of these proteins is presented. Formation of homotrimers, and a high degree of thermal and chemical stability reveals their true nature as \emph{bona fide} tail fibres, both of which are hall marks of previously resolved tail fibre proteins. Furthermore, their secondary structures, as determined by circular dichroism, follow similar patterns of minimal $\alpha$-helix, instead being dominated by $\beta$-sheets and turn motifs.

\subsection{Improved annotations lend confidence to a chimeric fibre structure}
PVCs harbouring tail fibres has been proposed since their discovery, however annotations were typically mixed, often with low confidence scores, leaving much to be desired in terms of a conclusive explanation. A couple of promising models were able to be obtained in Chapter \ref{structbioinfo}, but the variability and putative chimerism meant that the simulation quality varied enormously, leaving questionable confidence in even the better models. Quality of hits for both identified domains of the tail fibres have improved in recent years, with regions of T4-like and Adenovirus-like homology now able to be demarcated quite clearly. What seemed like a spurious annotation some years ago (Adenovirus homology within a bacterial operon), now appears increasingly plausible. Of course, to say the PVC fibres resemble Adenoviral domains is probably incorrect, and it is simply that the Protein DataBank is saturated with resolved structures from phage and clinically relevant human pathogenic viruses. However, its clear that PVC tail fibres are not purely phage like. Hopefully their structures will be able to be resolved fully in future work, as promising results were achieved in crystallography attempts. If so, these proteins would represent the first natural chimeras of a prokaryotic and eukaryotic viral motif known.

\subsection{PVC tail fibres potentially interact with cell surface proteins/sugars}
The chapter showed some promising preliminary results for establishing potential binding partners for these fibres, and hopefully this will identify the cell and tissue specificities that the PVCs have when exerting their effects. Preliminary data (albeit for a single PVC fibre) suggested a specificity for galactose- and sialic acid-bearing sugars, as well as components of the desmosome, a cell surface and inter-cell junction complex. The identification of sugar binding activity in particular is intriguing, as phage fibres are known to associate with lipopolysaccharides, and a number of studies have performed similar array based experiments to good effect. In a separate study not discussed here, the tail fibres were also shown not to interact with lipids, meaning that the sugar and protein targets identified are all the more likely to be worth pursuing.


\section{Chapter \ref{regulation}: PVC regulation is complex and heterogenous}
\subsection{PVCs are difficult to clone!}
One of the early goals of this thesis was to develop and test new methods for heterologous expression of the PVCs in a robust manner. Unfortunately, these methods are either not appropriate or require considerably more optimisation. In the mean time, success has been achieved by cloning using classical piece-wise restriction based methods. In future, a mixture of PCR/Gibson based assembly, combined with the higher efficiency of restriction based cloning likely remains the optimal way of producing these constructs.

\subsection{Antitermination and operon polarity suppression is implicated in PVC production}
Identification of a surprisingly canonical \emph{ops} site (surprising, because \emph{Photorhabdus} rarely likes to do much of anything canonically), suggests a role for anti-termination in the expression of PVC operons. Several long operons, particularly those which yield products destined for the extracellular environment, are known to be regulated by anti-termination mechanisms, often through the RfaH transcription factor, and the \emph{ops} site. Antitermination enhances the transcription of long operons, by enhancing RNA polymerase processivity for approximately an additional 20 kb downstream. Further structural investigation showed that the Afp actually harbours an extended \emph{ops} motif known as the JUMPStart sequence, forming a very particular secondary structure, but this is not found in PVCs. This may explain why Hurst \emph{et al.} were able to tightly control the expression of Afps on the native pADAP plasmid, by manipulating AnfA1, but no equivalent protein (other than a chromosomal copy of RfaH) can be found in \emph{Photorhabdus} - there may actually be subtle differences to their regulation. Unusually, subsequent cloning in the lab has shown that functional PVCs can be produced without the presence of the natural PVC 5' UTR altogether, including the \emph{ops} sequence; indicating that RfaH-like proteins/\emph{ops} sites, are not \emph{required} for PVC production, but may serve to fettle the expression in an as-yet-undetermined manner.

\subsection{PVCs production is subject to significant population heterogeneity}
Reporter microscopy has finally answered a long-pondered question. Namely, whether PVCs were produced by (potentially sacrificial) subpopulations of a culture. It seems that not only is this true, but the degree to which the operons are produced (and assuming this gives rise to expressed proteins/functional PVCs), differs greatly amongst the cells. Since at least part of an active \emph{Photorhabdus} infection is sacrificial, as food for the nematode, it was proposed that these cells may have to lyse to release active PVCs - given their size. No evidence of lysis has been detected in these assays or any others in the lab to date, suggesting that, miraculously, \emph{Photorhabdus} has also found a mechanism of releasing these complexes, whereas heterologously expressing \emph{E. coli} cultures suffer massive viability reduction. However, heterogeneity in expression remains. Not only this, but there are stark differences in the pattern of expression of different PVC operons. Combined with legacy RNAseq data from different inducing conditions, this implies that PVCs are indeed deployed under different and specific circumstances, by subsets of the population.



