\pagestyle{IHA-fancy-style}
\lhead{\textsf{Chapter 7}}
\rhead{\textsf{Discussion}}


\chapter{Discussion}\label{Discussion}

Major talking points for discussion:

Size of PVC lumen - effector folding
Mosaicism of paralogs
Population density/disparity of expression
Tail fibre binding candidates/role of diversity etc

Applicability to biotech


Future directions:
	- tail fibre crystallisation
	- further binding studies
	- further in depth study of population heterogeneity/response to different stimuli
	

Remaining PVC mysteries:

Role of ATPase unknown, its conservation may not be totally attributable its role in the PVCs if it performs other cellular roles.
Role of 2 proteins with enzymatic potential (PVC7 and 12) remains unknown
PVC 6 has no good homologies.
Suggest role of lysozyme as split from vgrG? Not sure why its retained
Role of various Lumt proteins not well understood
Significance of 2/3 outer sheath proteins still not understood (one maybe a cap?)

Also firmly identified Lumt (and Pnf) as unique in various ways. Conversely, Cif operons look to be a mainstay (as well as Unit4?)


This thesis suggests:
 - PVC5 as collar not tube
 - Fibres as chimeric anti eukaryotic proteins
 - PVC11 is a major baseplate component and possible mounting site of the tail fibres
 - PVC19 is the tube assembly initiator (though T6SS like spikes (which PVCs appear to be) have been shown to spontaenously assemble from vgrG).
 - PVC10 is a PAAR spike protein
 - PVC14 is similar to tape measure
 - 
 
 
 Pnf operons appear to be the most T4 like.
 
 
PVCs are T6SS like in their spikes, T4 like in their inner tubes, Pyocin like in outer tubes.

PVCs are obviously most like Afps, but both are therefore hybrids of T6SS and T4. No single structure appears to dominate the homologies.